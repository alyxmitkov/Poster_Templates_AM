\documentclass[final]{beamer}
\usepackage[scale=1.0,size=a0,orientation=landscape]{beamerposter}
\usepackage{graphicx,xcolor,booktabs}

% Define UC Merced colors
\definecolor{ucmercedblue}{RGB}{15,45,82}
\definecolor{ucmercedgold}{RGB}{218,169,0}

% Set color theme for blocks and structure
\setbeamercolor{structure}{fg=ucmercedblue}
\setbeamercolor{block title}{fg=white,bg=ucmercedblue}
\setbeamercolor{block body}{fg=black,bg=white}
\setbeamercolor{block title example}{fg=black,bg=ucmercedgold}
\setbeamercolor{block body example}{fg=black,bg=white}
\setbeamercolor{block title alerted}{fg=white,bg=ucmercedgold!85!black}
\setbeamercolor{block body alerted}{fg=black,bg=white}

\begin{document}

% Calculate 3 column layout (3 columns + 4 gaps = full width)
\newlength{\sepwidth}
\newlength{\colwidth}
\setlength{\sepwidth}{0.025\paperwidth}   % 2.5% page width separators
\setlength{\colwidth}{0.30\paperwidth}    % 30% page width per column
\newcommand{\separatorcolumn}{\begin{column}{\sepwidth}\end{column}}

\begin{frame}[t]

  % --- Top Banner ---
  \begin{beamercolorbox}[wd=\paperwidth, colsep=0.5cm]{block title}
    \begin{minipage}[c]{0.15\textwidth}
      \includegraphics[height=8cm]{LeftLogo.eps}
    \end{minipage}\hfill
   \begin{minipage}[c]{0.7\textwidth}
  \centering
  {\Huge \textbf{\textcolor{ucmercedgold}{Quantum Control in Driven Many-Body Systems}}}\\[2ex]
  {\LARGE \underline{Alyx Mitkov\textsuperscript{1}}, Lin Tian\textsuperscript{2}}\\[1ex]
  {\large
    \textsuperscript{1}Department of Chemistry and Biochemistry, University of California, Merced\\
    \textsuperscript{2}Department of Physics, University of California, Merced\\
    5200 North Lake Road, Merced, CA 95343, USA
  }
\end{minipage}
\hfill
    \begin{minipage}[c]{0.15\textwidth}
      \flushright \includegraphics[height=8cm]{RightLogo.eps}
    \end{minipage}
  \end{beamercolorbox}

  \vspace{1cm}  % space between banner and content

  \begin{columns}[t]

    \begin{column}{\colwidth}
      \begin{block}{Introduction}
        This is an introduction section. Provide some background or motivation for your project here. 
        Keep the text concise and engaging so that readers quickly grasp the context.
      \end{block}

      \begin{block}{Methods}
        Describe the methodology or approach. You can use bullet points for clarity:
        \begin{itemize}
          \item First, we did \textit{emphasis on procedure}.
          \item Next, we implemented \textit{technique or analysis}.
          \item Finally, we evaluated the outcomes.
        \end{itemize}
      \end{block}

      \begin{exampleblock}{Case Study}
        This example block highlights a specific case study or example result. 
        Its header is gold to distinguish it from the regular sections.
      \end{exampleblock}
    \end{column}

    \separatorcolumn

    \begin{column}{\colwidth}
      \begin{block}{Results}
        Key results are presented here. For instance, you might include a figure (see Figure 1):
        \begin{center}
          \includegraphics[width=0.8\columnwidth]{SampleFigure.png}\\
          {\small \textbf{Figure 1:} Caption describing the sample figure.}
        \end{center}

        You can also include a table (Table 1):
        \begin{center}
          \begin{tabular}{lrr}
            \toprule
            \textbf{Metric} & \textbf{Method A} & \textbf{Method B}\\
            \midrule
            Accuracy  & 85\% & 88\% \\
            Precision & 80\% & 85\% \\
            Recall    & 78\% & 82\% \\
            \bottomrule
          \end{tabular}\\
          {\small \textbf{Table 1:} Performance comparison between Method A and Method B.}
        \end{center}
      \end{block}

      \begin{alertblock}{Important Note}
        This alert block can be used for important takeaways or critical information. 
        Its header uses a darker gold with white text to attract attention. 
      \end{alertblock}
    \end{column}

    \separatorcolumn

    \begin{column}{\colwidth}
      \begin{block}{Conclusion}
        Summarize your conclusions here. Highlight the main points and the significance of your results. 
        You might also mention future work or applications in a sentence or two.
      \end{block}

      \begin{block}{References}
        \small  % smaller text for references
        [1] A. Author et al. (2023). *Title of a relevant paper or source*. Journal/Conference.\\
        [2] B. Author (2022). *Another reference title*. Publisher/Organization.\\
        [3] C. Author & D. Author (2021). *Additional reference if needed*. Journal/Book.\\
      \end{block}
    \end{column}

  \end{columns}
\end{frame}

\end{document}
